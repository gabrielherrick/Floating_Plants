
\documentclass[12pt]{amsart}
\usepackage{geometry} % see geometry.pdf on how to lay out the page. There's lots.
\geometry{letterpaper} % or letter or a5paper or ... etc
% \geometry{landscape} % rotated page geometry

% See the ``Article customise'' template for come common customisations

\title{}
\author{}
\date{} % delete this line to display the current date

%%% BEGIN DOCUMENT
\begin{document}

\maketitle


\section{Opening}
\subsection{Attention Getter} 
Lakes and streams can be healthy, and they can be unhealthy. (Lake Hollingsworth)

\subsection{Need}
NNC and LVI rely on correlation between plants and reference waters. We need causal links between nutrient levels and shifts in population and community dynamics of plants. 

\subsection{Task}
Look for differences among species in response to increasing nutrient levels from within NNC to exceeding NNC

\subsection{Main message}
Population models can be used to predict responses to nutrient levels. 


\subsection{Preview}
I will show you
\begin{enumerate}
\item Population growth models as a tool for prediction
\item Response of three floating plants to nutrients
\item Sneak peek at two-species models
\end{enumerate}

\section{Body}

Exceeding NNC changes population dynamics of floating plants

\begin{enumerate}
\item Population growth models
\item Lemna
\item Salvinia
\item Azolla
\item Sneak peek
\end{enumerate}


\section{Closing}

\subsection{Review}
\begin{itemize}
\item I have shown you that population growth models can be used to predict shifts in dynamics at appropriate nutrient levels. 
\item These three species of floating plants all show strong evidence of logistic growth. 
\item Two species models will help predict community dynamics
\end{itemize}
\subsection{Conclusion}
Aquatic plant population dynamics are sensitive to nutrient levels surrounding the NNC, but a lot more research is needed to support using plants as indicators of water quality. 

\subsection{Close}







\end{document}






